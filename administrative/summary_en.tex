\documentclass[a4paper,11pt]{article}

\usepackage[english]{babel}
\usepackage{xcolor}
\usepackage{tabularx}
\usepackage{lastpage}
\usepackage{fancyhdr}
\fancypagestyle{summaryen}{%
    \renewcommand{\headrulewidth}{0pt}
    \renewcommand{\headheight}{24pt}
    \fancyhead[C]{\small\color{gray} Dissertation summary \\ Krol, L.~R. (2020) \textit{Neuroadaptive Technology: Concepts, Tools, and Validations.}}
    \fancyfoot[C]{\small\color{gray} Page~\thepage~of~\pageref{LastPage}}
}
\pagestyle{summaryen}

\begin{document}


\section*{Summary}

This dissertation presents conceptual, methodological, and experimental advances in the field of neuroadaptive technology. Neuroadaptive technology refers to the category of technology that uses implicit input obtained from brain activity using a passive brain-computer interface in order to adapt itself, e.g. to enable implicit control or implicit interaction. Implicit input refers to any input obtained by a receiver that was not intended as such by the sender. Neuroadaptive technology thus detects naturally-occurring brain activity that was not intended for communication or control, and uses it to enable novel human-computer interaction paradigms. 

Part~I provides conceptual frameworks to unify previous works and guide future research. Chapter~1 reviews existing applications of passive brain-computer interfacing and suggests their level of interactivity to be a key parameter, ranging from mental state assessment, through open- and closed-loop adaptation, to forms of automated or intelligent adaptation. Systems in this latter category necessarily possess some autonomy to guide the interaction according to their own goals. Chapter~2 explains how this autonomy can be used for cognitive probing: a method in which the technology deliberately elicits a brain response from the user in order to learn from it. This allows neuroadaptive technology to exploit the fact that human brains automatically respond to the events they perceive. The gathered information can be used to further optimise the interaction, but can also be used in adverse ways. Chapter~2 therefore discusses a number of technological and ethical issues surrounding this method.

Part~II introduces two tools to help validate some core methods related to neuroadaptive technology. Chapter~3 describes SEREEGA (Simulating Event-Related EEG Activity), a free and open source toolbox to simulate event-related electroencephalographic (EEG) activity. Because simulated data has a known ground truth, it can be used to evaluate and validate analysis methods. SEREEGA covers and extends the vast majority of past and present-day EEG simulation approaches. Chapter~4 then uses such simulated data to validate a classifier visualisation method. This method allows a number of commonly-used classification algorithms to be visualised in a virtual brain, revealing which (cortical) areas the classifier focused on. This provides important insight to validate the classifier itself and neuroadaptive technology more broadly. It also provides a new classifier-based analysis method for neuroscientific research in general.

Part~III presents two experimental studies illustrating the technology described in Part~I, using the methods from Part~II. Chapter~5 demonstrates how neuroadaptive technology can be used to enable implicit cursor control using cognitive probing. By repeatedly eliciting brain responses to initially random cursor movements, and classifying these responses as reflecting either positive or negative interpretations of each movement, a computer can gradually reinforce the cursor to move in the direction desired by the observer. Importantly, the observer need not be aware of this happening. Chapter~6 presents additional analyses of this paradigm, revealing that brain activity elicited by the cursor movements can indeed reflect internal, subjective interpretations. These experiments highlight both the potential benefits and the potential risks addressed in Part~I.

\vfill

\begin{tabularx}{\textwidth}{XXX}
& & \hrule Klaus Gramann \\
\end{tabularx}

\vspace{5cm}

\end{document}
